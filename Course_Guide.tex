% Options for packages loaded elsewhere
\PassOptionsToPackage{unicode}{hyperref}
\PassOptionsToPackage{hyphens}{url}
%
\documentclass[
]{article}
\usepackage{lmodern}
\usepackage{amssymb,amsmath}
\usepackage{ifxetex,ifluatex}
\ifnum 0\ifxetex 1\fi\ifluatex 1\fi=0 % if pdftex
  \usepackage[T1]{fontenc}
  \usepackage[utf8]{inputenc}
  \usepackage{textcomp} % provide euro and other symbols
\else % if luatex or xetex
  \usepackage{unicode-math}
  \defaultfontfeatures{Scale=MatchLowercase}
  \defaultfontfeatures[\rmfamily]{Ligatures=TeX,Scale=1}
\fi
% Use upquote if available, for straight quotes in verbatim environments
\IfFileExists{upquote.sty}{\usepackage{upquote}}{}
\IfFileExists{microtype.sty}{% use microtype if available
  \usepackage[]{microtype}
  \UseMicrotypeSet[protrusion]{basicmath} % disable protrusion for tt fonts
}{}
\makeatletter
\@ifundefined{KOMAClassName}{% if non-KOMA class
  \IfFileExists{parskip.sty}{%
    \usepackage{parskip}
  }{% else
    \setlength{\parindent}{0pt}
    \setlength{\parskip}{6pt plus 2pt minus 1pt}}
}{% if KOMA class
  \KOMAoptions{parskip=half}}
\makeatother
\usepackage{xcolor}
\IfFileExists{xurl.sty}{\usepackage{xurl}}{} % add URL line breaks if available
\IfFileExists{bookmark.sty}{\usepackage{bookmark}}{\usepackage{hyperref}}
\hypersetup{
  pdftitle={Universität Salzburg \textbar{} Brad Mackay},
  pdfauthor={Winter Semester 20/21},
  hidelinks,
  pdfcreator={LaTeX via pandoc}}
\urlstyle{same} % disable monospaced font for URLs
\usepackage[margin=1in]{geometry}
\usepackage{longtable,booktabs}
% Correct order of tables after \paragraph or \subparagraph
\usepackage{etoolbox}
\makeatletter
\patchcmd\longtable{\par}{\if@noskipsec\mbox{}\fi\par}{}{}
\makeatother
% Allow footnotes in longtable head/foot
\IfFileExists{footnotehyper.sty}{\usepackage{footnotehyper}}{\usepackage{footnote}}
\makesavenoteenv{longtable}
\usepackage{graphicx}
\makeatletter
\def\maxwidth{\ifdim\Gin@nat@width>\linewidth\linewidth\else\Gin@nat@width\fi}
\def\maxheight{\ifdim\Gin@nat@height>\textheight\textheight\else\Gin@nat@height\fi}
\makeatother
% Scale images if necessary, so that they will not overflow the page
% margins by default, and it is still possible to overwrite the defaults
% using explicit options in \includegraphics[width, height, ...]{}
\setkeys{Gin}{width=\maxwidth,height=\maxheight,keepaspectratio}
% Set default figure placement to htbp
\makeatletter
\def\fps@figure{htbp}
\makeatother
\setlength{\emergencystretch}{3em} % prevent overfull lines
\providecommand{\tightlist}{%
  \setlength{\itemsep}{0pt}\setlength{\parskip}{0pt}}
\setcounter{secnumdepth}{-\maxdimen} % remove section numbering
\ifluatex
  \usepackage{selnolig}  % disable illegal ligatures
\fi
\newlength{\cslhangindent}
\setlength{\cslhangindent}{1.5em}
\newenvironment{cslreferences}%
  {\setlength{\parindent}{0pt}%
  \everypar{\setlength{\hangindent}{\cslhangindent}}\ignorespaces}%
  {\par}

\title{Universität Salzburg \textbar{} Brad Mackay}
\author{Winter Semester 20/21}
\date{}

\begin{document}
\maketitle

\hypertarget{course-description}{%
\subsection{Course Description}\label{course-description}}

The course aims to introduce students to a variety of different `Global
Englishes' beyond the British or American varieties with which they may
already be familiar. English is spoken (to some extent) by an estimated
2,000,000,000 people (Crystal 2004) and is the official language in over
80 sovereign and non-sovereign states. We will examine the historical
and political contexts that have led to the global spread of English and
we will take a closer look at the ways English is used, and how it has
developed, in different contexts around the world.

The course is split into two main thematic blocks: first, we will
consider the main theoretical concepts which have been used to study the
spread of English as a global language, and in the second block, we will
be exploring these theories through a series of case studies, looking at
a variety of English from a different context each week. We will use
video, audio and written text to explore the variation in locally marked
phonetic, syntactic and lexical features.

\hypertarget{learning-outcomes}{%
\subsection{Learning Outcomes}\label{learning-outcomes}}

By the end of this course, students will:

-- Have developed an understanding of the political and historical
contexts which have led to the spread of English as a lingua franca.

-- Have been introduced to the main concepts behind the spread of
English as a global language.

-- Be able to discuss the following points in relation to different
English-speaking contexts (see Schneider 2011 :xvii):

\begin{itemize}
\tightlist
\item
  The \textbf{reasons} behind the spread of English (why has it
  spread?).
\item
  The \textbf{process} behind the spread of English (how has it
  spread?).
\item
  The \textbf{results} of this spread (i.e.~where do you find English
  and what role does it play in different societies?).
\item
  The \textbf{properties} of these varieties (i.e.~the phonetic,
  morphosyntactic, lexical variation found in different contexts).
\item
  The \textbf{consequences} of the spread of English (how does the
  presence of English affect people's everyday lives? How do people in
  different contexts view the use of English?).
\end{itemize}

\hypertarget{resources}{%
\subsection{Resources}\label{resources}}

The readings for each session are available on Blackboard. Please make
sure you do the required reading \textbf{before} the session. There will
also be extra suggested reading for each session, this is not a
requirement but is a good starting point if you want to find out more
about a particular topic.Seminar worksheets will be made available on
Blackboard before each session.

\hypertarget{corpus-of-global-web-based-english-glowbe}{%
\subsubsection{Corpus of Global Web-Based English
(GloWbE)}\label{corpus-of-global-web-based-english-glowbe}}

We will be using the GloWbE corpus (Davies and Fuchs 2015). You can
access this corpus for free with your university email account, you just
need to register an account with your university email address first,
this can be done by clicking
\href{https://www.english-corpora.org/glowbe/}{here}, or by going to
\url{https://www.english-corpora.org/glowbe/}

\hypertarget{assessment}{%
\subsection{Assessment}\label{assessment}}

\hypertarget{reading-15}\label{reading-15}}

\hypertarget{midterm-25}\label{midterm-25}}

\hypertarget{presentation-20}\label{presentation-20}}

\hypertarget{final-project-40}\label{final-project-40}}

\begin{yellowbox}

\begin{center}

\textbf{Reminder}

\end{center}

Assessment is to be undertaken individually. All suspected cases of
plagiarism and collusion will be reported and investigated.You must
attend class consistently. You must attend at least 80\% of the
sessions, excessive lateness will be considered absence.

\end{yellowbox}

\hypertarget{feedback}{%
\subsection{Feedback}\label{feedback}}

You will receive written feedback on the midterm, your presentation and
your final assignment. The reading assignments on Blackboard are graded
automatically and you will receive a grade after completing each test.
If you would like any further feedback or to discuss your feedback
further please come to my office hours or email me to arrange a meeting.

\hypertarget{suggested-reading}{%
\subsection{Suggested Reading}\label{suggested-reading}}

If you are accessing this guide online, you can click on the pink links
and they should take you directly to the text. If you are not accessing
this guide from a university computer, you may need to use the
university VPN. For information on accessing information from home click
\href{https://www.uni-salzburg.at/index.php?id=32331}{here} I would
highly recommend that everyone read
\href{https://www.cambridge.org/core/services/aop-cambridge-core/content/view/30FFC5253F465905D75CDFF1C1363AE3/S0047404517000562a.pdf/unsettling_race_and_language_toward_a_raciolinguistic_perspective.pdf}{Unsettling
race and language: Toward a raciolinguistic perspective}
({\textbf{???}})

\hypertarget{syllabus}{%
\subsection{Syllabus}\label{syllabus}}

\begin{longtable}[]{@{}llll@{}}
\toprule
Session & Date & Topic & Preparation\tabularnewline
\midrule
\endhead
1 & 08/10 & Introduction: Who owns English? Attitudes to English &
None\tabularnewline
- & No class this week & &\tabularnewline
2 & 22/10 & The Spread of English & Global Englishes
p35-56\tabularnewline
3 & 29/10 & Pidgins and Creoles & Global Englishes p35-56\tabularnewline
4 & 05/11 & &\tabularnewline
5 & 12/11 & &\tabularnewline
6 & 19/11 & &\tabularnewline
7 & 26/11 & &\tabularnewline
8 & 03/12 & &\tabularnewline
9 & 10/12 & &\tabularnewline
10 & 17/12 & &\tabularnewline
11 & 14/01 & &\tabularnewline
12 & 21/01 & &\tabularnewline
13 & 29/01 & &\tabularnewline
\bottomrule
\end{longtable}

\hypertarget{references}{%
\subsection*{References}\label{references}}
\addcontentsline{toc}{subsection}{References}

\hypertarget{refs}{}
\begin{cslreferences}
\leavevmode\hypertarget{ref-davies2015expanding}{}%
Davies, Mark, and Robert Fuchs. 2015. ``Expanding Horizons in the Study
of World Englishes with the 1.9 Billion Word Global Web-Based English
Corpus (Glowbe).'' \emph{English World-Wide} 36 (1): 1--28.

\leavevmode\hypertarget{ref-schneider2011english}{}%
Schneider, Edgar W. 2011. \emph{English Around the World: An
Introduction}. Cambridge University Press.
\end{cslreferences}

\end{document}
